\documentclass[12pt,letterpaper]{article}
\usepackage{fullpage}
\usepackage[top=2cm, bottom=4.5cm, left=2.5cm, right=2.5cm]{geometry}
\usepackage{amsmath,amsthm,amsfonts,amssymb,amscd}
\usepackage{lastpage}
\usepackage{enumerate}
\usepackage{fancyhdr}
\usepackage{mathrsfs}
\usepackage{xcolor}
\usepackage{graphicx}
\usepackage{listings}
\usepackage{hyperref}
\usepackage{mathtools}
\usepackage{xfrac}

\hypersetup{%
  colorlinks=true,
  linkcolor=blue,
  linkbordercolor={0 0 1}
}
\linespread{1.1}
 
\renewcommand\lstlistingname{Algorithm}
\renewcommand\lstlistlistingname{Algorithms}
\def\lstlistingautorefname{Alg.}

\lstdefinestyle{Python}{
    language        = Python,
    frame           = lines, 
    basicstyle      = \footnotesize,
    keywordstyle    = \color{blue},
    stringstyle     = \color{green},
    commentstyle    = \color{red}\ttfamily
}

\setlength{\parindent}{0.0in}
\setlength{\parskip}{0.05in}

% Edit these as appropriate
\newcommand\course{ConvexOptimization}
\newcommand\hwnumber{5}                  % <-- homework number
\newcommand\NetIDa{SUN Yilin}           % <-- NetID of person #1
\newcommand\NetIDb{520030910361}           % <-- NetID of person #2 (Comment this line out for problem sets)

\pagestyle{fancyplain}
\headheight 35pt
\lhead{\NetIDa}
\lhead{\NetIDa\\\NetIDb}                 % <-- Comment this line out for problem sets (make sure you are person #1)
\chead{\textbf{\Large Homework \hwnumber}}
\rhead{\course \\ \today}
\lfoot{}
\cfoot{}
\rfoot{\small\thepage}
\headsep 1.5em

\begin{document}

\section{}
By the first-order optimality condition we know that the solution $\boldsymbol{x}^*$ to the given problem must satisfy that $\nabla f(\boldsymbol{x}^*)^T(\boldsymbol{x}-\boldsymbol{x}^*)\geq0,\forall\boldsymbol{x}\in\Bar{B}$. By taking the gradient of $f$ we know that the inequality is equivalent to $(\boldsymbol{x}^*-\boldsymbol{x}_0)^T(\boldsymbol{x}-\boldsymbol{x}^*)\geq0$. We already know that this inequality is also the condition of the projection of point $\boldsymbol{x}_0$ onto $\Bar{B}$. Now to show that the projection of $\boldsymbol{x}_0\notin\Bar{B}$ is $\frac{\boldsymbol{x}_0}{||\boldsymbol{x}_0||}$, we only need to show that $\frac{\boldsymbol{x}_0}{||\boldsymbol{x}_0||}$ satisfies $(\frac{\boldsymbol{x}_0}{||\boldsymbol{x}_0||}-\boldsymbol{x}_0)^T(\boldsymbol{x}-\frac{\boldsymbol{x}_0}{||\boldsymbol{x}_0||})\geq0$.\\
Then by transformation, that is equivalent to show $\boldsymbol{x}_0^T\boldsymbol{x}\leq||\boldsymbol{x}_0||$. Then by applying $H\ddot{o}lder's$ $Inequality$ we get 
$\boldsymbol{x}_0^T\boldsymbol{x}=\sum_{i=1}^{n}x_{0_{i}}x_i\leq\sum_{i=1}^{n}|x_{0_{i}}x_i|\leq||\boldsymbol{x}_0||\cdot||\boldsymbol{x}||\leq||\boldsymbol{x}_0||$. So $\frac{\boldsymbol{x}_0}{||\boldsymbol{x}_0||}$ does satisfy $(\frac{\boldsymbol{x}_0}{||\boldsymbol{x}_0||}-\boldsymbol{x}_0)^T(\boldsymbol{x}-\frac{\boldsymbol{x}_0}{||\boldsymbol{x}_0||})\geq0$. So it is the projection of $\boldsymbol{x}_0\notin\Bar{B}$.

\section{}
Below is the feasible set of the problem. Note that the area is directed by the arrows.
\begin{figure}[htbp]
    \centering
    \includegraphics[width=0.6\textwidth,height=0.5\textwidth]{Fset.png}
    \caption{\label{fig:fset} Feasible Set for Problem 2}
\end{figure}
\subsection*{(a)}
Graphically, the set of optimal solutions is a single point, i.e. $S=\{(x_1,x_2)=(0.4,0.2)\}$. And the optimal value is $f=0.6$. Below is the graph where we can see the optimal solution, i.e. the intersection of the blue line and the Feasible Set.
\begin{figure}[htbp]
    \centering
    \includegraphics[width=0.4\textwidth,height=0.3\textwidth]{2a.png}
    \caption{Set of Optimal Solutions for 2(a)}
\end{figure}

\subsection*{(b)}
The set of optimal solutions is where either $x_1$ or $x_2$ goes to infinity, and the optimal value is just minus infinity. We cannot represent the optimal solutions with a graph but we know where it is.
\subsection*{(c)}
Graphically, the set of optimal solutions is a set of points, i.e. $S=\{(x_1,x_2)|x_1=0,x_2\geq1\}$. And the optimal value is $f=0$. Below is the graph where we can see the optimal solutions, i.e. the blue line above the point $(x_1,x_2)=(0,1)$.
\begin{figure}[htbp]
    \centering
    \includegraphics[width=0.4\textwidth,height=0.3\textwidth]{2c.png}
    \caption{\label{fig:2c} Set of Optimal Solutions for 2(c)}
\end{figure}
\subsection*{CVXPY Results for (a)-(e)}
For the optimal variables, the first is $x_1$, the second is $x_2$.
\begin{figure}[htbp]
    \centering
    \includegraphics[width=1\textwidth,height=0.8\textwidth]{prob2rst.jpeg}
    \caption{\label{fig:cvxpy} CVXPY Results for Problem 2}
\end{figure}

\section{}
\subsection*{(a)}
We introduce a new variable $\boldsymbol{t}\in\mathbb{R}^m$ and let $-\boldsymbol{t}\leq\boldsymbol{Ax}-\boldsymbol{b}\leq\boldsymbol{t}$, where the inequality is defined elementwise. And note that for the constraint $||\boldsymbol{x}||_{\infty}\leq1$, it is equivalent to $-\boldsymbol{1}\leq\boldsymbol{x}\leq\boldsymbol{1}$(also elementwise). Then the original problem can be transformed into
$$\min_{\boldsymbol{x},\boldsymbol{t}}\boldsymbol{1}^T\boldsymbol{t}$$
$$s.t.\quad  -\boldsymbol{1}\leq\boldsymbol{x}\leq\boldsymbol{1}$$
$$\qquad \qquad\quad-\boldsymbol{t}\leq\boldsymbol{Ax}-\boldsymbol{b}\leq\boldsymbol{t}$$
\subsection*{(b)}
Note that the optimal variable is given in form of a row vector, whose transpose is the actual answer.
\begin{figure}[htbp]
    \centering
    \includegraphics[width=0.8\textwidth,height=0.16\textwidth]{prob3rst.png}
    \caption{\label{fig:cvxpy3} CVXPY Results for Problem 3(b)}
\end{figure}
\subsection*{(c)}
\begin{figure}[htbp]
    \centering
    \includegraphics[width=0.8\textwidth,height=0.16\textwidth]{prob3lp.png}
    \caption{\label{fig:cvxpy3lp} CVXPY Results for Problem 3(c)}
\end{figure}
\section{}
\subsection*{(a)}
Since $\boldsymbol{X}$ has full column rank, $\boldsymbol{w}^*$ is simply $(\boldsymbol{X}^T\boldsymbol{X})^{-1}\boldsymbol{X}^T\boldsymbol{y}$, which is $(1.5,2)^T$. And the optimal value is 4.
\subsection*{(b)(c)}
\begin{figure}[htbp]
    \centering
    \includegraphics[width=0.8\textwidth,height=0.6\textwidth]{prob4.jpeg}
    \caption{\label{fig:cvxpy4} CVXPY Results for Problem 4}
\end{figure}
Lasso with $t=1$ has different solution, and the solution has zero component. But Lasso with $t=10$ has the same solution with no zero component. Ridge with $t=1$ has different solution with no zero component, while with $t=100$ the solution is the same with no zero compenent.
\end{document}
