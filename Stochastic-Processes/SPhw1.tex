\documentclass[12pt,letterpaper]{article}
\usepackage{fullpage}
\usepackage[top=2cm, bottom=4.5cm, left=2.5cm, right=2.5cm]{geometry}
\usepackage{amsmath,amsthm,amsfonts,amssymb,amscd}
\usepackage{lastpage}
\usepackage{enumerate}
\usepackage{fancyhdr}
\usepackage{mathrsfs}
\usepackage{xcolor}
\usepackage{graphicx}
\usepackage{listings}
\usepackage{hyperref}
\usepackage{mathtools}
\usepackage{xfrac}


\hypersetup{%
  colorlinks=true,
  linkcolor=blue,
  linkbordercolor={0 0 1}
}
\linespread{1.1}
 
\renewcommand\lstlistingname{Algorithm}
\renewcommand\lstlistlistingname{Algorithms}
\def\lstlistingautorefname{Alg.}


\lstdefinestyle{Python}{
    language        = Python,
    frame           = lines, 
    basicstyle      = \footnotesize,
    keywordstyle    = \color{blue},
    stringstyle     = \color{green},
    commentstyle    = \color{red}\ttfamily
}

\setlength{\parindent}{0.0in}
\setlength{\parskip}{0.05in}

% Edit these as appropriate
\newcommand\course{Stochastic Processes}
\newcommand\hwnumber{1}                  % <-- homework number
\newcommand\NetIDa{SUN Yilin}           % <-- NetID of person #1
\newcommand\NetIDb{520030910361}           % <-- NetID of person #2 (Comment this line out for problem sets)

\pagestyle{fancyplain}
\headheight 35pt
\lhead{\NetIDa}
\lhead{\NetIDa\\\NetIDb}                 % <-- Comment this line out for problem sets (make sure you are person #1)
\chead{\textbf{\Large Homework \hwnumber}}
\rhead{\course \\ \today}
\lfoot{}
\cfoot{}
\rfoot{\small\thepage}
\headsep 1.5em

\begin{document}

\section{Probability Space of Tossing Coins}
\subsection{}
$\forall n\in\mathbb{N}$ and $\forall s\in\{0,1\}^n$, $\exists\omega\in\Omega$ such that $\omega_i=s_i$, $\forall i\in n$. So $\forall n\in\mathbb{N},\forall s$, $C_s\neq\emptyset$.\\
$\forall n\in\mathbb{N}$, for any two sets $C_{s_{i}}$ and $C_{s_{j}}$ such that $i\neq j$, $\exists k\in [n]$ such that $s_{ik}\neq s_{jk}$. So $C_{s_{i}}$ and $C_{s_{j}}$ are disjoint.\\
$\forall n\in\mathbb{N}$ and $\forall\omega\in\Omega$, $\exists s\in\{0,1\}^n$ such that $\omega_i=s_i$, $\forall i\in n$, which means $\exists s\in \{0,1\}^n$ such that $\omega\in C_s$. So $\cup_{s\in\{0,1\}^n}C_s=\Omega$.\\
So $\forall n\in\mathbb{N}$, the collection $\{C_s\}$ forms a partition of $\Omega$.
\subsection{}
For every $n\in\mathbb{N}$, since $\mathcal{F}_n$ is generated by $\{C_s\}$, $C_s\in\mathcal{F}_n$. From what we have proven above we know that $\{C_s\}$ forms a partition of $\Omega$, so any of their unions are distinctive to each other and their complements are simply their unions formed by other sets.
So the cardinality of $\mathcal{F}_n$ is $$\sum_{i=0}^{n}C_{2^n}^{i}=2^{2^n}$$
Also note that the cardinality of $2^{\{0,1\}^n}$ is simply $2^{|\{0,1\}^n|}=2^{2^n}$, so $\mathcal{F}_n$ and $2^{\{0,1\}^n}$ are equinumerous. So there exists a bijection between them.
\subsection{} 
For any $n\in\mathbb{N}$, $\forall C_s\in \mathcal{F}_n$, $\exists s_1,s_2\in \{0,1\}^{n+1}$ such that $C_s=C_{s_1}\cup C_{s_2}$. 
By definition, $C_{s_1},C_{s_2}\in\mathcal{F}_{n+1}$. Since $\mathcal{F}_{n+1}$ is a $\sigma$-algebra, $C_{s_1}\cup C_{s_2}\in\mathcal{F}_{n+1}$, meaning that $C_s\in\mathcal{F}_{n+1}$.\\
Note that $\mathcal{F}_{n}$ and $\mathcal{F}_{n+1}$ are both $\sigma$-algebra, so for the unions and complements of $C_s$ which are in $\mathcal{F}_{n}$, they must also be in $\mathcal{F}_{n+1}$. So $\mathcal{F}_{n}\subset\mathcal{F}_{n+1}$. Note that $\mathcal{F}_{n}$ and $\mathcal{F}_{n+1}$ are not equinumerous as we have proven above, meaning that $\mathcal{F}_{n}\neq\mathcal{F}_{n+1}$. 
So $\mathcal{F}_{n}\subsetneq\mathcal{F}_{n+1}$. So the sequence of sets $\mathcal{F}_1,\mathcal{F}_2,\dots$ is increasing.
\subsection{}
$\forall A\in\mathcal{F}_{\infty}$, $\exists n\in\mathbb{N}$ such that $A\in\mathcal{F}_{n}$. Since $\mathcal{F}_{n}$ is a $\sigma$-algebra, $A^C\in\mathcal{F}_{n}$. So $A^C\in\mathcal{F}_{\infty}$.\\
$\forall A,B\in\mathcal{F}_{\infty}$, $\exists m,n\in\mathbb{N}$ such that $A\in\mathcal{F}_n,B\in\mathcal{F}_m$. without loss of generality, let $m\geq n$. From the last question we know $A\in\mathcal{F}_m$ as well. Since $\mathcal{F}_m$ is a $\sigma$-algebra, $A\cup B\in\mathcal{F}_m$, meaning that $A\cup B\in\mathcal{F}_{\infty}$. So $\mathcal{F}_{\infty}$ is an algebra.\\
$\forall\omega\in\Omega$, consider the set $\{\omega\}\in 2^{\Omega}$. It is clear that $\{\omega\}\notin\mathcal{F}_{n},\forall n\in\mathbb{N}$. So $\{\omega\}\notin\mathcal{F}_{\infty}$. So $\mathcal{F}_{\infty}\neq 2^{\Omega}$. 
\subsection{}
$\forall\omega\in\Omega$, $\{\omega\}=\cap_{i=1}^{\infty}C_{s_i}$ where $s_i\in\{0,1\}^i$. It can also be expressed as $\{\omega\}=(\cup_{i=1}^{\infty}C_{s_i}^C)^C$ by DeMorgan's law.
Since $C_{s_i}\in\mathcal{F}_{\infty}$, $C_{s_i}\in\sigma(\mathcal{F}_{\infty})$. Since $\sigma(\mathcal{F}_{\infty})$ is a $\sigma$-algebra, $\{\omega\}=(\cup_{i=1}^{\infty}C_{s_i}^C)^C\in\sigma(\mathcal{F}_{\infty})$.\\
Note that we have proven $\{\omega\}\notin\mathcal{F}_{\infty}$ in the last question, so $\{\omega\}\in\sigma(\mathcal{F}_{\infty})\backslash \mathcal{F}_{\infty}$.
\subsection{}
$\forall A\in\mathcal{F}_{\infty}$, $\exists n\in\mathbb{N}$ such that $A\in\mathcal{F}_{n}$. Since $\{C_s\}$ is a partition of $\Omega$, $\exists C_{s_i}, i\in[k]$ such that $A\subset\cup_{i=1}^{k}C_{s_i}$ and $\forall C_{s_i},\exists a\in A$ such that $a\in C_{s_i}$. If $A\neq\cup_{i=1}^{k}C_{s_i}$, $\exists a\in\cup_{i=1}^kC_{s_i},a\notin A$, so $\exists C_s$ such that $a\in C_s,a\notin A$, i.e. $a\in C_s\backslash A$. That is to say, $\exists C_s\backslash A$ which satisfies that $C_s\backslash A\neq\emptyset, C_s\backslash A\neq C_s$. \\
Now notice that $A\in\mathcal{F}_n,C_s\in\mathcal{F}_n$, so $A\cup C_s^C\in\mathcal{F}_n$. So $C_s\backslash A=(A\cup C_s^C)^C\in\mathcal{F}_n$, which is in contradiction with the fact that $\mathcal{F}_n$ is the $\sigma$-algebra generated by $\{C_s\}$. So the assumption that $A\neq\cup_{i=1}^k C_{s_i}$ is not true, it follows that $A=\cup_{i=1}^k C_{s_i}$.\\
Now by the existence of $n$ we know that there exists a smallest $n_0$ and $k_0$ accordingly. Suppose we have derived the value $\frac{k_0}{2^{n_0}}$, we now prove that $\forall n>n_0$, $\frac{k}{2^n}=\frac{k_0}{2^{n_0}}$ by induction. Consider $n_0+1$, we have shown in problem 1.3 that $\forall C_{s_i}$ such that $A=\cup_{i=1}^{k_0} C_{s_i}$, there exists exactly two $C_{s_1},C_{s_2}$ where $s_1,s_2\in \{0,1\}^{n_0+1}$ such that $C_{s_i}=C_{s_1}\cup C_{s_2}$. So the number $k$ for $n_0+1$ is actually $2k_0$. So the number for $n_0+1$ is actually $\frac{2k_0}{2^{n_0+1}}$, which is equal to $\frac{k_0}{2^{n_0}}$. This holds true for any $n_0>n$, so we can say the value only depends on $A$.
\subsection{}
\subsubsection{My original trial of proof}
Below is my original trial of proof before our teacher stated that we should use Caratheodory's extension Theorem to prove.\\
Firstly we can construct a probability measure $P$ where $\forall A\in\mathcal{F}_{\infty}$, $P(A)=\frac{k}{2^n}$ as defined above; $\forall \{\omega\}\in\mathcal{B}(\Omega)\backslash\mathcal{F}_{\infty}$, $P(\{\omega\})=0$; $P(\Omega)=1$. As for other events in $\mathcal{B}(\Omega)$, the probability can be calculated by the countable additivity. Surely such a probability measure does exist.\\
Then we claim that a probability measure with respect to $\{\omega\}$ must be $P(\{\omega\})=0$, hence $P$ is unique. Consider any $\{\omega\}$, it can be expressed as $\{\omega\}=\cap_{i=1}^{\infty}C_{s_i}$ where $s_i\in\{0,1\}^{i}$ and $\forall i>j, C_{s_i}\subset C_{s_j}$. Thus by the continuity of probability measures $P(\{\omega\})=P(\cap_{i=1}^{\infty}C_{s_i})=\lim_{i\to \infty}P(C_{s_i})=\lim_{i\to\infty}\frac{1}{2^i}=0$. So the probability measure is unique. 
\subsubsection{Proof by Caratheodory's extension Theorem}
Caratheodory's extension Theorem says that any pre-measure defined on a given ring $R$ of subsets of a given set $\Omega$ can be extended to a measure on the $\sigma$-algebra generated by $R$, and this extension is unique if the pre-measure is $\sigma$-finite. Consequently, any pre-measure on a ring containing all intervals of real numbers can be extended to the Borel algebra of the set of real numbers.\\
So here our $\sigma$-algebra is $\sigma(\mathcal{F}_{\infty})$. Our pre-measure defined above on $\mathcal{F}_{\infty}$ hence can be extended to a measure on the $\sigma$-algebra generated by $\mathcal{F}_{\infty}$, i.e. $\sigma(\mathcal{F}_{\infty})$. And clearly our definition of the pre-measure is  a finite number, hence $\sigma$-finite, so our extension is unique. So by Caratheodory's extension Theorem there exists a unique probability measure satisfying the restrictions given by this problem.
\subsection{}
Toss a fair coin infinitely many times, let $X$ be the number of trials until the first Head(or Tail if you want), then the distribution of $X$ is geometric distribution with parameter $\frac{1}{2}$ in the probability space we have constructed above.
\section{Conditional Expectations}
\subsection{}
Since $f$ is a measurable function, $\forall a\in\mathbb{R}$, notice that singletons in $\mathbb{R}$ are also Borel Sets, $A=f^{-1}(a)$ must be a Borel Set. The random variable $X$ is $\sigma(X)$-measurable itself, so for any Borel Set $A$, $X^{-1}(A)\in\sigma(X)$. So $\forall a\in\mathbb{R}, f(X)^{-1}(a)\in\sigma(X)$, which means $f(X)$ is $\sigma(X)$-measurable.
\subsection{}
Since $\sigma(Y)=\sigma(Y')$, $\forall \omega\in\Omega, Y^{-1}(Y(\omega))=Y'^{-1}(Y'(\omega))$, denoted by $A$. So $\forall\omega\in\Omega$, $\mathbb{E}(X|Y=Y(\omega))=\mathbb{E}(X|Y'=Y'(\omega))=\mathbb{E}(X|A)$.
\subsection{}
$\mathbb{E}(X|\mathcal{F})$ should be defined as a random variable from $\Omega$ to $\mathbb{R}$. $\forall\omega\in\Omega$, $\mathbb{E}(X|\mathcal{F(\omega)})=\mathbb{E}(X|A)$ where $A$ is the element satisfying $A\in\mathcal{F},\omega\in A$ with the smallest cardinality.   
\subsection{}
Firstly we can find such sets $A_i, i\in [n]$ in $\mathcal{F}_2$ which are not supersets of any other elements in $\mathcal{F}_2$, i.e. $\mathcal{F}_2$ is the $\sigma$-algebra generated by $\{A_i\}$. Then $\forall A\in\mathcal{F}_1$, $\exists A_i\in\mathcal{F}_2,i\in [k]$ such that $A=\cup_{i=1}^{k}A_i$.\\
Now $\forall\omega\in\Omega$, by definition of last problem, $\mathbb{E}(X|\mathcal{F}_1(\omega))=\mathbb{E}(X|A)=\sum_{x}xPr(X=x|A)$ where $A\in\mathcal{F}_1$, $\omega\in A$ and $A$ has the smallest cardinality. Now we compute the value of $\mathbb{E}(\mathbb{E}(X|\mathcal{F}_1)|\mathcal{F}_2)$ and $\mathbb{E}(\mathbb{E}(X|\mathcal{F}_2)|\mathcal{F}_1)$ at $\omega$ respectively.\\
\begin{align}
\mathbb{E}(\mathbb{E}(X|\mathcal{F}_2)|\mathcal{F}_1(\omega))&=\sum_{x}xPr(\mathbb{E}(X|\mathcal{F}_2)=x|A)\\
&=\sum_{i=1}^{n}\mathbb{E}(X|A_i)Pr(A_i|A)\\
&=\sum_{i=1}^{n}\sum_{x}xPr(X=x|A_i)Pr(Ai|A)\
&=\sum_{A_i\subset A}\sum_{x}xPr(X=x|A_i)Pr(Ai|A)\\
&=\sum_{A_i\in A}\sum_{x}xPr(X=x|A_i)\frac{Pr(Ai)}{Pr(A)}\\
&=\sum_{A_i\in A}\sum_{x}x\frac{Pr(X=x\wedge A_i)}{Pr(A_i)}\frac{Pr(Ai)}{Pr(A)}\\
&=\sum_xx\sum_{A_i\subset A}\frac{Pr(X=x\wedge A_i)}{Pr(A)}\\
&=\sum_xxPr(X=x|A)\\
&=\mathbb{E}(X|\mathcal{F}_1(\omega))
\end{align}

\begin{align}
\mathbb{E}(\mathbb{E}(X|\mathcal{F}_1)|\mathcal{F}_2(\omega))&=\sum_{x}xPr(\mathbb{E}(X|\mathcal{F}_1)=x|A_i), \omega\in A_i, A_i\in\mathcal{F}_2\\
&=\sum_{A\subset\mathcal{F}_1}\mathbb{E}(X|A)Pr(A|A_i)\\
&=\mathbb{E}(X|A)Pr(A|A_i), \omega\in A, \omega\in A_i, A\in\mathcal{F}_1, A_i\in\mathcal{F}_2\\
&=\mathbb{E}(X|A)*1\\
&=\mathbb{E}(X|\mathcal{F}_1(\omega))
\end{align}

\section*{Acknowledgements}
The idea of constructing the probability measure is from Dr. Krishna Jagannathan's Lecture Notes in course EE5110: Probability Foundations for Electrical Engineers. In Lecture 8, he described the basic ideas of this construction without showing the details. I tried to 
understand it and express in my own words in problem 1.7
\end{document}
