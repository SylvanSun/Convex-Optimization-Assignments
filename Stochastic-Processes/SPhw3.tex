\documentclass[12pt,letterpaper]{article}
\usepackage{fullpage}
\usepackage[top=2cm, bottom=4.5cm, left=2.5cm, right=2.5cm]{geometry}
\usepackage{amsmath,amsthm,amsfonts,amssymb,amscd}
\usepackage{lastpage}
\usepackage{enumerate}
\usepackage{fancyhdr}
\usepackage{mathrsfs}
\usepackage{xcolor}
\usepackage{graphicx}
\usepackage{listings}
\usepackage{hyperref}
\usepackage{mathtools}
\usepackage{xfrac}
\usepackage{algorithm}
\usepackage[noend]{algpseudocode}

\hypersetup{%
  colorlinks=true,
  linkcolor=blue,
  linkbordercolor={0 0 1}
}
\linespread{1.1}
 
\renewcommand\lstlistingname{Algorithm}
\renewcommand\lstlistlistingname{Algorithms}
\def\lstlistingautorefname{Alg.}

\lstdefinestyle{Python}{
    language        = Python,
    frame           = lines, 
    basicstyle      = \footnotesize,
    keywordstyle    = \color{blue},
    stringstyle     = \color{green},
    commentstyle    = \color{red}\ttfamily
}

\setlength{\parindent}{0.0in}
\setlength{\parskip}{0.05in}

% Edit these as appropriate
\newcommand\course{Stochastic Processes}
\newcommand\hwnumber{3}                
\newcommand\NetIDa{SUN Yilin}         
\newcommand\NetIDb{520030910361}           

\pagestyle{fancyplain}
\headheight 35pt
\lhead{\NetIDa}
\lhead{\NetIDa\\\NetIDb}           
\chead{\textbf{\Large Homework \hwnumber}}
\rhead{\course \\ \today}
\lfoot{}
\cfoot{}
\rfoot{\small\thepage}
\headsep 1.5em

\begin{document}
\section{FTMC for countably infinite chains}
\subsection{}
To prove this, we only need to show $[F]+[A]+[I]$ implies $[PR]+[A]+[I]$.\newline
To show this, we only need to show $[F]+[I]$ implies $[PR]$, i.e.
a finite and irreducible markov chain is positive recurrent.\newline
\newline
Firstly, for a finite markov chain, there exists at least one recurrent state.
If not, all the states in this markov chain can only be visited finitely many times.
But this finite markov chain only contains finite states,
so when we visit infinitely many times this cannot be true.\\
Then if the markov chain is also irreducible, since recurrence is a class property,
the existence of a single recurrent state means this markov chain is recurrent.\\
\newline
Now we only need to show a finite irreducible recurrent markov chain is positive recurrent.
Similar to our proof in lecture notes, we know that positive recurrence and null recurrence are both class properties.
Then if the recurrent markov chain contains a null recurrent state, it is null recurrent itself.
Since $\sum_{j\in S}P_{ij}^{n}=1$ and there are only finite number of states,
it is impossible that $\lim_{n\to\infty}P_{ij}^{n}=0$ for all $j\in S$.
Thus this markov chain cannot be null recurrent. Then it can only be positive recurrent.\\
\newline
To sum up, a finite irreducible markov chain must be positive recurrent.
So the proposition in this problem about these two implications is correct.

\newpage
\section{Acknowledgements}
\subsection{}
The idea of proving a finite irreducible markov chain is from lecture 5 of STAT253/317, University of Chicago.
\end{document}
