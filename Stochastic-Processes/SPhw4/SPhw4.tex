\documentclass[12pt,letterpaper]{article}
\usepackage{fullpage}
\usepackage[top=2cm, bottom=4.5cm, left=2.5cm, right=2.5cm]{geometry}
\usepackage{amsmath,amsthm,amsfonts,amssymb,amscd}
\usepackage{lastpage}
\usepackage{enumerate}
\usepackage{fancyhdr}
\usepackage{mathrsfs}
\usepackage{xcolor}
\usepackage{graphicx}
\usepackage{listings}
\usepackage{hyperref}
\usepackage{mathtools}
\usepackage{xfrac}
\usepackage{bbm}

\hypersetup{
  colorlinks=true,
  linkcolor=blue,
  linkbordercolor={0 0 1}
}
\linespread{1.1}
 
\renewcommand\lstlistingname{Algorithm}
\renewcommand\lstlistlistingname{Algorithms}
\def\lstlistingautorefname{Alg.}


\lstdefinestyle{Python}{
    language        = Python,
    frame           = lines, 
    basicstyle      = \footnotesize,
    keywordstyle    = \color{blue},
    stringstyle     = \color{green},
    commentstyle    = \color{red}\ttfamily
}

\setlength{\parindent}{0.0in}
\setlength{\parskip}{0.05in}

\newcommand\course{Stochastic Processes}
\newcommand\hwnumber{4}
\newcommand\NetIDa{SUN Yilin}
\newcommand\NetIDb{520030910361}

\pagestyle{fancyplain}
\headheight 35pt
\lhead{\NetIDa}
\lhead{\NetIDa\\\NetIDb}
\chead{\textbf{\Large Homework \hwnumber}}
\rhead{\course \\ \today}
\lfoot{}
\cfoot{}
\rfoot{\small\thepage}
\headsep 1.5em

\begin{document}

\section{Doob's Martingale Inequality}
Consider $\tau=\arg\min_{t\leq n}\{X_t\geq\alpha\}$ or $\tau=n$ if $\forall 0\leq t\leq n, X_t<\alpha$.\\
Clearly $\tau$ is a stopping time because by our definition 
for any $t\geq 0$, $\mathbbm{1}[\tau\leq t]$ is $\mathcal{F}_t$-measurable.\\
Then denote event $X_{\tau}\geq\alpha$ by $A$,
denote event $\max_{0\leq t\leq n}X_t\geq\alpha$ by $B$.
We have $B\subset A$ because by our definition of stopping time $\tau$,
if $X_{\tau}\geq\alpha$,
then if must follows that $\exists k, 0\leq k\leq n$ such that $X_t\geq\alpha$,
hence $\max_{0\leq t\leq n}X_t\geq\alpha$.
We know that if $B\subset A$, then $\Pr(B)\leq\Pr(A)$.
This means that
$$\Pr \left[\max_{0\leq t\leq n}X_t\geq\alpha\right]\leq \Pr\left[X_{\tau}\geq\alpha\right]$$
Since $X_t\geq 0$, by applying the Markov Inequality we have
$$\Pr\left[X_{\tau}\geq\alpha\right]\leq \frac{\mathbb{E}\left[X_{\tau}\right]}{\alpha}$$
By our definition of $\tau$ we can easily see $\Pr[\tau\leq n]=1$,
which means that $\tau$ is bounded almost surely,
satisfying the first condition for Optional Stopping Theorem.
Then by OST we have
$$\mathbb{E}[X_{\tau}]=\mathbb{E}[{X_0}]$$
Adding up all the inequalities together we get 
$$\Pr \left[\max_{0\leq t\leq n}X_t\geq\alpha\right]\leq \frac{\mathbb{E}\left[X_{0}\right]}{\alpha}$$
which completes our proof.

\section{Biased One-dimensional Random Walk}
\subsection{}
\begin{align}
  \mathbb{E}(S_{t+1}|\overline{Z_{1,n}})
  &=\mathbb{E}(S_t+Z_{t+1}+2p-1|\overline{Z_{1,n}})\\
  &=S_t+2p-1+\mathbb{E}(Z_{t+1}|\overline{Z_{1,n}})\\
  &=S_t+2p-1+(-1)\cdot p+1\cdot(1-p)\\
  &=S_t
\end{align}
So $\{S_t\}$ is a martingale.
\subsection{}
\begin{align}
  \mathbb{E}(P_{t+1}|\overline{Z_{1,n}})
  &=\mathbb{E}((\frac{p}{1-p})^{X_t+Z_{t+1}}|\overline{Z_{1,n}})\\
  &=\mathbb{E}((\frac{p}{1-p})^{X_t}\cdot(\frac{p}{1-p})^{Z_{t+1}}|\overline{Z_{1,n}})\\
  &=P_t\cdot\mathbb{E}((\frac{p}{1-p})^{Z_{t+1}}|\overline{Z_{1,n}})\\
  &=P_t\cdot(\frac{p}{1-p}\cdot(1-p)+\frac{1-p}{p}\cdot p)\\
  &=P_t
\end{align}
So $\{P_t\}$ is a martingale.
\subsection{}
If $p=\frac{1}{2}$, we have shown in class that $\mathbb{E}(\tau)=ab$ in class.
Here we only show the case where $p\neq\frac{1}{2}$.\\
Clearly, $\Pr(\tau<\infty)$ also holds when $p\neq\frac{1}{2}$. 
$|X_t|$ is bounded, so $|P_t|$ is bounded,
indicating that $\{P_t\}$ satisfies the second condition of OST.
Also, $\mathbb{E}(|S_{t+1}-S_{t}||\mathcal{F}_t)\leq 2p+1$,
indicating that $\{S_t\}$ satisfies the third condition of OST.
So we can apply the Optional Stopping theorem on them and thus we have
$\mathbb{E}(S_{\tau})=\mathbb{E}(S_{1})$ and $\mathbb{E}(P_{\tau})=\mathbb{E}(P_{1})$.\\
Denote $\Pr$(ending at $-a$) by $P_a$, $\Pr$(ending at $b$) by $P_b$.
From $\mathbb{E}(S_{\tau})=\mathbb{E}(S_{1})=0$ we have 
$$\mathbb{E}(\tau)\cdot(2p-1)=aP_a-bP_b$$
From $\mathbb{E}(P_{\tau})=\mathbb{E}(P_{1})=1$ we have
$$\left(\frac{p}{1-p}\right)^{-a}P_a+\left(\frac{p}{1-p}\right)^{b}P_b=1$$
By the latter equation we can calculate $P_a$ and $P_b$,
$$P_a=\frac{1-\left(\frac{p}{1-p}\right)^b}{\left(\frac{p}{1-p}\right)^{-a}-\left(\frac{p}{1-p}\right)^b}\qquad
P_b=\frac{\left(\frac{p}{1-p}\right)^{-a}-1}{\left(\frac{p}{1-p}\right)^{-a}-\left(\frac{p}{1-p}\right)^b}$$
By putting these two results to the former equation we get
$$\mathbb{E}(\tau)=\frac{(a-b)-a\left(\frac{p}{1-p}\right)^b+b\left(\frac{p}{1-p}\right)^{-a}}
{(2p-1)\left[\left(\frac{p}{1-p}\right)^{-a}-\left(\frac{p}{1-p}\right)^b\right])} \quad (p\neq\frac{1}{2})$$

\section{Longest Common Subsequence}

\subsection{}

\subsection{}

\end{document}
