\documentclass[12pt,letterpaper]{article}
\usepackage{fullpage}
\usepackage[top=2cm, bottom=4.5cm, left=2.5cm, right=2.5cm]{geometry}
\usepackage{amsmath,amsthm,amsfonts,amssymb,amscd}
\usepackage{lastpage}
\usepackage{enumerate}
\usepackage{fancyhdr}
\usepackage{mathrsfs}
\usepackage{xcolor}
\usepackage{graphicx}
\usepackage{listings}
\usepackage{hyperref}
\usepackage{mathtools}
\usepackage{xfrac}
\usepackage{algorithm}
\usepackage[noend]{algpseudocode}

\hypersetup{
  colorlinks=true,
  linkcolor=blue,
  linkbordercolor={0 0 1}
}
\linespread{1.1}
 
\renewcommand\lstlistingname{Algorithm}
\renewcommand\lstlistlistingname{Algorithms}
\def\lstlistingautorefname{Alg.}


\lstdefinestyle{Python}{
    language        = Python,
    frame           = lines, 
    basicstyle      = \footnotesize,
    keywordstyle    = \color{blue},
    stringstyle     = \color{green},
    commentstyle    = \color{red}\ttfamily
}

\setlength{\parindent}{0.0in}
\setlength{\parskip}{0.05in}

\newcommand\course{Algorithms}
\newcommand\hwnumber{4}
\newcommand\NetIDa{SUN Yilin}
\newcommand\NetIDb{520030910361}

\pagestyle{fancyplain}
\headheight 35pt
\lhead{\NetIDa}
\lhead{\NetIDa\\\NetIDb}
\chead{\textbf{\Large Homework \hwnumber}}
\rhead{\course \\ \today}
\lfoot{}
\cfoot{}
\rfoot{\small\thepage}
\headsep 1.5em

\begin{document}

\section{}
\subsection{}
The idea is to scan the array from left to right and update the maximum revenue meanwhile.
See algorithm below for details.
\begin{algorithm}
  \caption{Max Revenue-1,1($\boldsymbol{a}$)}\label{alg:mr11}
  \begin{algorithmic}[1]
  \Procedure{Max Revenue-1,1}{$\boldsymbol{a}$}:\Comment{$\boldsymbol{a}=\{a_1,a_2,a_3,\dots,a_n\}$}
  \State result$\gets0$, sum$\gets0$, $i\gets1$ \Comment{sum is our current optimal to maintain}
  \While{$i<n+1$}
  \State sum$\gets$sum+$a_i$
  \If{sum$>$result}
  \State result$\gets$sum\Comment{Update the optimal subsequence}
  \EndIf
  \If{sum$<$0}
  \State sum$\gets$0\Comment{Discard the bad subsequence we do not want}
  \EndIf
  \State $i\gets i+1$
  \EndWhile
  \State \textbf{return} result
  \EndProcedure
  \end{algorithmic}
\end{algorithm}\\
Since our algorithm only requires us to scan the array once,
clearly the time complexity is $O(n)$.

\newpage
\subsection{}
The idea is still scanning the array from left to right once,
but we will use an array $\boldsymbol{s}[n]$ to store the maximum revenue
of a subsequence ending at position $i,i\in[n]$.\\
First we initialize $s_1,s_2,s_3,\dots,s_L$.
Then for each element $a_i,i\in[n]$, 
we need to update $R-L+1$ elements in $\boldsymbol{s}[n]$,
which are $s_{i+L},s_{i+L+1},\dots,s_{i+R}$.
The initialize rule and update rule are given in the algorithm below.
\begin{algorithm}
  \caption{Max Revenue($L,R,\boldsymbol{a}$)}\label{alg:mrlrNsquare}
  \begin{algorithmic}[1]
  \Procedure{Max Revenue}{$L,R,\boldsymbol{a}$}:\Comment{$\boldsymbol{a}=\{a_1,a_2,a_3,\dots,a_n\}$}
  \State result$\gets0$, $i\gets1$ $s_i\gets 0,\forall i\in[n]$\Comment{$\boldsymbol{s}$ is described above}
  \While{$i<L+1$}
  \If{$a_i>0$}
  \State $s_i\gets a_i$\Comment{Initialization for the first $L$ elements}
  \EndIf
  \If{$a_i>$result}
  \State result$\gets a_i$\Comment{Update the result}
  \EndIf
  \State $i\gets i+1$
  \EndWhile
  \State $i\gets1$
  \While{$i+L<n+1$}
  \State step$\gets L$
  \While{$i+$step$<n+1$}
  \If{$s_i+a[i+$step$]>s[i+$step$]$}
  \State $s[i+$step$]=s_i+a[i+$step$]$\Comment{update the optimal result at $i$}
  \EndIf
  \If{$s[i+$step$]>$result}
  \State result$\gets s[i+$step$]$ 
  \EndIf
  \State step$\gets$step+1
  \EndWhile
  \State $i\gets i+1$ 
  \EndWhile
  \State \textbf{return} result
  \EndProcedure
  \end{algorithmic}
\end{algorithm}\\
We need to do $R-L+1$ updates at each round and there are $n$ rounds in total,
so the time complexity is $O((R-L+1)n)$.
So with the difference between $L$ and $R$ approaching $n$,
our algorithm actually becomes $O(n^2)$.

\newpage
\subsection{}
We still need to scan the array from left to right once,
but instead of updating $R-L+1$ elements at each iteration,
we use a better strategy.\\
For each $a_i$,
we look for the largest $s_j,i-R\leq j\leq i-L$
and use this largerst result to update $s_i$.
Then by our algorithm in class,
we can find all the largest $s_j$ for our current $a_i$ with only $O(n)$ time.
A detailed algorithm is given below.
\begin{algorithm}
  \caption{Max Revenue($L,R,\boldsymbol{a}$)}\label{alg:mrlrn}
  \begin{algorithmic}[1]
  \Procedure{Max Revenue}{$L,R,\boldsymbol{a}$}:\Comment{$\boldsymbol{a}=\{a_1,a_2,a_3,\dots,a_n\}$}
  \State result$\gets0$, $i\gets1$ $s_i\gets 0,\forall i\in[n]$ 
  \While{$i<L+1$}
  \If{$a_i>0$}
  \State $s_i\gets a_i$\Comment{Initialization is the same as algorithm 2}
  \EndIf
  \If{$a_i>$result}
  \State result$\gets a_i$\Comment{Update the result}
  \EndIf
  \State $i\gets i+1$
  \EndWhile
  \State $i\gets L+1$
  \While{$i<n+1$}
  \State max$\gets$ k-Largest($\boldsymbol{s},i-R,i-L$)\Comment{Algorithm in class to find max in $O(1)$ time}
  \If{max+$a_i>s_i$}
  \State $s_i\gets$max+$a_i$
  \EndIf
  \If{$s_i>$result}
  \State result$\gets s_i$
  \EndIf
  \State $i\gets i+1$
  \EndWhile
  \State \textbf{return} result
  \EndProcedure
  \end{algorithmic}
\end{algorithm}\\
The reason why this algorithm is faster is that
by using the k-Largest algorithm, 
we do not have to update $R-L+1$ times each round.
We only need to look up the largest result before a certain element 
and the look up process is $O(1)$ for each element.
As a result, the total running time of our algorithm can be reduced to $O(n)$.


\end{document}
