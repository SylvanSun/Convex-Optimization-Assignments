\documentclass[12pt,letterpaper]{article}
\usepackage{fullpage}
\usepackage[top=2cm, bottom=4.5cm, left=2.5cm, right=2.5cm]{geometry}
\usepackage{amsmath,amsthm,amsfonts,amssymb,amscd}
\usepackage{lastpage}
\usepackage{enumerate}
\usepackage{fancyhdr}
\usepackage{mathrsfs}
\usepackage{xcolor}
\usepackage{graphicx}
\usepackage{listings}
\usepackage{hyperref}
\usepackage{mathtools}
\usepackage{xfrac}
\usepackage{algorithm}
\usepackage[noend]{algpseudocode}
\usepackage{bbm}

\hypersetup{
  colorlinks=true,
  linkcolor=blue,
  linkbordercolor={0 0 1}
}
\linespread{1.1}
 
\renewcommand\lstlistingname{Algorithm}
\renewcommand\lstlistlistingname{Algorithms}
\def\lstlistingautorefname{Alg.}


\lstdefinestyle{Python}{
    language        = Python,
    frame           = lines, 
    basicstyle      = \footnotesize,
    keywordstyle    = \color{blue},
    stringstyle     = \color{green},
    commentstyle    = \color{red}\ttfamily
}

\setlength{\parindent}{0.0in}
\setlength{\parskip}{0.05in}

\newcommand\course{Algorithms}
\newcommand\hwnumber{5}
\newcommand\NetIDa{SUN Yilin}
\newcommand\NetIDb{520030910361}

\pagestyle{fancyplain}
\headheight 35pt
\lhead{\NetIDa}
\lhead{\NetIDa\\\NetIDb}
\chead{\textbf{\Large Homework \hwnumber}}
\rhead{\course \\ \today}
\lfoot{}
\cfoot{}
\rfoot{\small\thepage}
\headsep 1.5em

\begin{document}

\section{}


\newpage
\section{}
\subsection{}
We first construct a weighted directed graph $H$ from $G$.
Then we can find maxflow on graph $H$ to see whether there is a perfect matching in $G$.\\
Firstly, we let all the edges in $G$ point from $V_1$ to $V_2$ in $H$,
and we assign them a weight of $+\infty$.\\
Then we add a source vertex $s$ and 
add an edge with weight $1$ from $s$ to all vertices in $V_1$.\\
Finally we add a sink vertex $t$ and 
add an edge with weight $1$ from all vertices in $V_2$ to $t$.\\
Now we have constructed the graph $H$,
we can run the Network Flow algorithm to find a maxflow of $H$.
If the maxflow of $H$, denoted by $f$,
turns out to be $f=|V_1|=|V_2|$,
then there exists a perfect matching in $G$.
And the perfect matching contains exactly the edges in $G$ used 
by us to construct the maxflow of $H$.
\subsection{}
\textbf{Proof of Necessity:}\\
This part is very straightforward.\\
Suppose there is a perfect matching $M$ from $V_1$ to $V_2$.
Then for any $S\subset V_1$, 
for every vertex $v\in S$,
there is an edge in $M$ connecting $v$ to a vertex in $V_2$.
This means that there are at least as many vertices in $V_2$ that are neighbors of vertices in $V_1$
as there are vertices in $V_1$.\\
That is to say, for any $S\subset V_1$,
$|N(S)|\geq|S|$.\\
\textbf{Proof of Sufficiency:}\\
Following the hint, we would like to prove by showing that the mincut 
of the graph we construct is exactly $|V_1|$(or $|V_2|$ if you like).\\
Firstly, any mincut can only contain the edges in $H$ which are not in $G$
because we assign edges in $G$ with a weight of $+\infty$.\\
Then there is a cut with capacity $|V_1|$ 
if we make our $S$-$T$ cut to be $S=\{s\}$, the singleton.
So the capacity of mincut of $H$ is at most $|V_1|$.\\
Suppose there is another minimum $S$-$T$ cut where $S\backslash V_2=\{s\}\cup (V_1\backslash V_1')$
and $T\backslash V_1=\{t\}\cup (V_2\backslash V_2')$,
since this is a mincut, 
there are no edges from $V_1\backslash V_1'$ to $V_2\backslash V_2'$.
This means that all the neighbors of $V_1\backslash V_1'$ must be in $V_2'$.
Then by the property that $|N(S)|\geq |S|$ for any $S\subset V_1$,
$|V_1\backslash V_1'|\leq |V_2'|$.\\
Finally the capacity of this cut is $|V_1'|+|V_2'|$,
$|V_1'|+|V_2'|\geq |V_1'|+|V_1\backslash V_1'|=|V_1|$.
This means that the capacity of any $S$-$T$ cut is at least $|V_1|$.
Then by our instance where $S=\{s\}$, the mincut is indeed $|V_1|$.
Finally by the Maxflow-Mincut Theorem, 
the maxflow of $H$ equals its mincut $|V_1|$,
which means that there is a perfect matching in $G$.
 
\newpage
\section{}
We prove by showing that the debt network $G$ can be equivalently transformed to 
another network $H$ which has $n-1$ edges at most.\\
In the original debt graph $G$, 
for any vertex $v\in V$, 
we define 
$$W(v)=\sum_{(u,v)\in E}w(u,v)-\sum_{(v,p)\in E}w(v,p)$$
$W(v)$ is the amount of money that $v$ owes other roommates or other roommates owe $v$.
We then divide $V$ into a partition $V=V_1\cup V_2\cup V_3$, 
where $\forall v\in V_1$, $W(v)>0$,
$\forall v\in V_2$, $W(v)<0$,
$\forall v\in V_3$, $W(v)=0$.\\
To sum up, $V_1$ is the set of all creditors, 
$V_2$ is the set of all debtors and $V_3$ is the set of all people who 
do not need to pay of receive money from anyone else.
In graph $H=(V',E')$ we do not have to consider vertices in $V_3$,
so $V'=V_1\cup V_2$.\\
Then we pick anyone from $V_2$ and call him the pivot $p$.
We construct the graph $H$ by the following method:\\
For all vertices $v$ in $V_1$, 
we add an edge $(p,v)$ with weight $W(v)$.\\
For all vertices $v$ in $V_2\backslash\{p\}$,
we add an edge $(v,p)$ with weight $W(v)$.\\
I.e., all debtors excluding pivot $p$ give money to $p$,
then $p$ gives money to all creditors(on behalf of all debtors).
So the total times it takes to carry out person-to-person payments is 
at most $|V_1|+|V_2|-1$, which is at most $n-1$.\\
\textbf{Proof of Correctness:}\\
The reason why $H$ and $G$ are equivalent when deciding the amount of payments
is that graph $H$ guarantees that every single person is 
receiving or giving out exactly the amount of money that he should collect 
or in debt.
So out construction of $H$ can indeed make sure that the debts are paid off at last.


\newpage
\section{Comments}
\subsection{}

\subsection{}

\subsection{}

\end{document}
