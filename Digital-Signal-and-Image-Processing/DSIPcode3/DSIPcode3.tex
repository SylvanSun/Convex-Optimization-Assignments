\documentclass[12pt,letterpaper]{article}
\usepackage{fullpage}
\usepackage[top=2cm, bottom=4.5cm, left=2.5cm, right=2.5cm]{geometry}
\usepackage{amsmath,amsthm,amsfonts,amssymb,amscd}
\usepackage{lastpage}
\usepackage{enumerate}
\usepackage{fancyhdr}
\usepackage{mathrsfs}
\usepackage{xcolor}
\usepackage{graphicx}
\usepackage{listings}
\usepackage{hyperref}
\usepackage{mathtools}
\usepackage{xfrac}


\hypersetup{%
  colorlinks=true,
  linkcolor=blue,
  linkbordercolor={0 0 1}
}
\linespread{1.1}
 
\renewcommand\lstlistingname{Algorithm}
\renewcommand\lstlistlistingname{Algorithms}
\def\lstlistingautorefname{Alg.}


\lstdefinestyle{Python}{
    language        = Python,
    frame           = lines, 
    basicstyle      = \footnotesize,
    keywordstyle    = \color{blue},
    stringstyle     = \color{green},
    commentstyle    = \color{red}\ttfamily
}

\setlength{\parindent}{0.0in}
\setlength{\parskip}{0.05in}

% Edit these as appropriate
\newcommand\course{DSIP}
\newcommand\hwnumber{3}                  % <-- homework number
\newcommand\NetIDa{SUN Yilin}           % <-- NetID of person #1
\newcommand\NetIDb{520030910361}           % <-- NetID of person #2 (Comment this line out for problem sets)

\pagestyle{fancyplain}
\headheight 35pt
\lhead{\NetIDa}
\lhead{\NetIDa\\\NetIDb}                 % <-- Comment this line out for problem sets (make sure you are person #1)
\chead{\textbf{\Large Programming \hwnumber}}
\rhead{\course \\ \today}
\lfoot{}
\cfoot{}
\rfoot{\small\thepage}
\headsep 1.5em

\begin{document}
The results of this assignment can be seen from the graph below.
Here I briefly explain how I implemented these four methods.
Note that I change the step size of how the sequence is increased from 4 to 2 to draw a better plot.
And the maximum sequence length is $2^{12}$.\\
\\
The definition method is nothing new. 
We can calculate the DFT directly by its formula and it takes the longest time as we expected.\\
\\
When calculating the time for matrix method, \textbf{I take it into account the time spent on generating the DFT matrix.} 
It turns out that if we do not use any technique and calculate the matrix directly, it is still time-consuming.
But we can speed up the process because the DFT matrix is symmetric. 
So it turns out that the matrix method is a lot faster than calculating by definition.\\
\\
When using the last two methods this process became so fast that we can barely see its plot in the same picture as the previous methods. 
As for the scale of our assignment, the time it takes for FFT whether using gpuArray or not will take no longer than 0.1 second.
Using gpuArray is the fastest method, but it is not a significant improvment over the plain fft method. 
Maybe the efficiency of gpuArray can be seen for a problem with larger scale.
Also note that gpuArray is actually a little bit slow when treating a sequence with small length.
\end{document}
