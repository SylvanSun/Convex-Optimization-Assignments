\documentclass[12pt,letterpaper]{article}
\usepackage{fullpage}
\usepackage[top=2cm, bottom=4.5cm, left=2.5cm, right=2.5cm]{geometry}
\usepackage{amsmath,amsthm,amsfonts,amssymb,amscd}
\usepackage{lastpage}
\usepackage{enumerate}
\usepackage{fancyhdr}
\usepackage{mathrsfs}
\usepackage{xcolor}
\usepackage{graphicx}
\usepackage{listings}
\usepackage{hyperref}
\usepackage{mathtools}
\usepackage{xfrac}


\hypersetup{%
  colorlinks=true,
  linkcolor=blue,
  linkbordercolor={0 0 1}
}
\linespread{1.1}
 
\renewcommand\lstlistingname{Algorithm}
\renewcommand\lstlistlistingname{Algorithms}
\def\lstlistingautorefname{Alg.}


\lstdefinestyle{Python}{
    language        = Python,
    frame           = lines, 
    basicstyle      = \footnotesize,
    keywordstyle    = \color{blue},
    stringstyle     = \color{green},
    commentstyle    = \color{red}\ttfamily
}

\setlength{\parindent}{0.0in}
\setlength{\parskip}{0.05in}

% Edit these as appropriate
\newcommand\course{DSIP}
\newcommand\hwnumber{2}                  % <-- homework number
\newcommand\NetIDa{SUN Yilin}           % <-- NetID of person #1
\newcommand\NetIDb{520030910361}           % <-- NetID of person #2 (Comment this line out for problem sets)

\pagestyle{fancyplain}
\headheight 35pt
\lhead{\NetIDa}
\lhead{\NetIDa\\\NetIDb}                 % <-- Comment this line out for problem sets (make sure you are person #1)
\chead{\textbf{\Large Programming \hwnumber}}
\rhead{\course \\ \today}
\lfoot{}
\cfoot{}
\rfoot{\small\thepage}
\headsep 1.5em

\begin{document}

\section{}
In this problem we set our sampling frequency as $f$ and our sampling interval is $T=1/f$. Originally I planned to set up two arrays to represent our original signal and sampling signal respectively. Then by a simple element-wise multiplication we get our sampled signal. But after the illustration of our teacher in class I decided not to use that method. Instead, I directly manipulate the array and represent the process of sampling by this manipulation. We plot the sampled signal in time domain and also show its features in frequency domain.\\
As we can see from the result, our sampled signal is exactly the original signal if we set sampling points properly, which means that we can get all information about the original signal in an ideal state.
\section{}
To do this we only need to left(or right if you want) shift our original signal by one bit in our array. Now after sampling we miss one bit of information. As we are using Dirac Comb as a sampling function, this means that we missed exactly one $\delta(t-nT)$. As we can see from the result, in the frequency domain the amplitude of our shifted signal is smaller than original signal. This is because the Fourier Transform of $\delta$ is a constant over the entire frequency domain. As we miss more $\delta$ functions, the amplitude naturally gets lower.
\section{}
We set proper parameters for the Butterworth filter and get a filtered signal. Now we can see after this finter process the amplitude in the frequency domain is actually larger. This is because after the filter process we generated s $sinc$ function, as we can see from the new plot in the time domain. Note that here we can also see the Gibbs peonomenon. 

\newpage 
\section{Formulas for these problems}
In this problem, although our signal to be sampled, the rectangle window, cannot be expressed as a continuous function, we can still simulate the sampling process. Let $T$ be our sampling interval, i.e. the multiplicative inverse of sampling frequency $f$, Then the sampling signal can be expressed as $S(t)=\sum_{l=1}^{10f}\delta(t-lT)$.
\subsection{}
The signal after sample is simply $x[n]=x(t)S(t)=\sum_{l=1}^{10f}x(nT)\delta(t-lT)=\sum_{l=1}^{10f}\delta(t-lT)$.\\
Hence after sampling $x[n]$ is a sequence of $\delta$ functions.\\
Then in frequency domain $\mathcal{F}(x)=\sum_{l=1}^{10f}exp(-jwlT)$.
\subsection{}
After shifting by the analysis above we miss one $\delta$ function. Hence $x'[n]=\sum_{l=1}^{10f-1}\delta(t-lT)$.\\
In frequency domain again we have $\mathcal{F}(x')=\sum_{l=1}^{10f-1}exp(-jwlT)$. And this explained why the amplitude is lower than the signal not shifted.
\subsection{}
Consider the single $\delta$ function that we miss. The reason that we missed this function is that it is an "off-grid" signal for our sampling function. 
\begin{align}
&\quad\int_{-\omega_0}^{\omega_0}exp(-j\omega(\tau))exp(j\omega nT)d\omega\\
&=\int_{-\omega_0}^{\omega_0}exp(-j\omega(\tau-nT))d\omega\\
&=\frac{1}{-j(\tau-nT)}\int_{-\omega_0}^{\omega_0}dexp(-j\omega(\tau-nT))\\
&=\frac{exp(-j\omega_0(\tau-nT))-exp(j\omega_0(\tau-nT))}{-j(\tau-nT)}\\
&=\frac{sin(\omega_0(\tau-nT))}{\tau-nT}\\
&=\omega_0sinc(\omega_0(\tau-nT))
\end{align}
Thus by low-pass filtering we recovered some of the information of the "off-grid" signal. And this explains what the filtered signal can have a amplitude between the original signal and shifted signal in frequency domain.
\end{document}
