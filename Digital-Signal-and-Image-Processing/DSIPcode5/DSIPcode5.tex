\documentclass[12pt,letterpaper]{article}
\usepackage{fullpage}
\usepackage[top=2cm, bottom=4.5cm, left=2.5cm, right=2.5cm]{geometry}
\usepackage{amsmath,amsthm,amsfonts,amssymb,amscd}
\usepackage{lastpage}
\usepackage{enumerate}
\usepackage{fancyhdr}
\usepackage{mathrsfs}
\usepackage{xcolor}
\usepackage{graphicx}
\usepackage{listings}
\usepackage{hyperref}
\usepackage{mathtools}
\usepackage{xfrac}
\usepackage{algorithm}
\usepackage[noend]{algpseudocode}
\usepackage{bbm}

\hypersetup{
  colorlinks=true,
  linkcolor=blue,
  linkbordercolor={0 0 1}
}
\linespread{1.1}
 
\renewcommand\lstlistingname{Algorithm}
\renewcommand\lstlistlistingname{Algorithms}
\def\lstlistingautorefname{Alg.}


\lstdefinestyle{Python}{
    language        = Python,
    frame           = lines, 
    basicstyle      = \footnotesize,
    keywordstyle    = \color{blue},
    stringstyle     = \color{green},
    commentstyle    = \color{red}\ttfamily
}

\setlength{\parindent}{0.0in}
\setlength{\parskip}{0.001cm}

\newcommand\course{DSIP}
\newcommand\hwnumber{5}
\newcommand\NetIDa{SUN Yilin}
\newcommand\NetIDb{520030910361}

\pagestyle{fancyplain}
\headheight 35pt
\lhead{\NetIDa}
\lhead{\NetIDa\\\NetIDb}
\chead{\textbf{\Large Homework \hwnumber}}
\rhead{\course \\ \today}
\lfoot{}
\cfoot{}
\rfoot{\small\thepage}
\headsep 0.1em

\begin{document}
\section{}
First we extract the Red channel of the original image,
then we use histeq function directly to see the results.\\
As it turns out,
the original image is dark,
which can also be revealed in its histogram.\\
After the histogram equalization process, 
it gains a higher contrast,
the grey-scale graph becomes brighter
and the histogram is more evenly distributed.

\section{}
The grey-scale image after exponential matching is more similar to the 
original image.\\
This is because,
as we can see from the histogram of the original image,
its original histogram is similar to an exponential distribution.
So as we do a histogram matching equalization,
if we use exponential distribution,
more features about the original image can be reserved.

\section{}
Here we use a uniform distribution to compare our result 
with the default settings of the histeq function.
By generating a histogram of uniformly distributed vatiables 
and letting out histogram match the uniform histogram,
we managed to do such an equalization.\\
As it turns out,
here the result is more similar to our result in question 1.

\section{}
By using the cat function we generate a new image from the equalized RGB channels.

\section{}
The problem of method in question 4 is that it carries out 
the equalization process respectively and 
then put the results of different channels together to generate a new image.
There might be two underlying problems.\\
First is that the RGB channels might have a different histogram distribution,
second is that their histogram might not be evenly distributed.
Here in our problem I checked that the RGB channels indeed follow a similar distribution,
so our major problem is that we should not directly use histeq function.
Instead, we should try to match the histogram to a exponential distribution histogram,
then the result will be much more similar to our original graph, 
which can be seen in the graph attached below. 

\end{document}
