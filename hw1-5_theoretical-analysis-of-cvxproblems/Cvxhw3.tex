\documentclass[12pt,letterpaper]{article}
\usepackage{fullpage}
\usepackage[top=2cm, bottom=4.5cm, left=2.5cm, right=2.5cm]{geometry}
\usepackage{amsmath,amsthm,amsfonts,amssymb,amscd}
\usepackage{lastpage}
\usepackage{enumerate}
\usepackage{fancyhdr}
\usepackage{mathrsfs}
\usepackage{xcolor}
\usepackage{graphicx}
\usepackage{listings}
\usepackage{hyperref}
\usepackage{mathtools}
\usepackage{xfrac}

\hypersetup{%
  colorlinks=true,
  linkcolor=blue,
  linkbordercolor={0 0 1}
}
\linespread{1.1}
 
\renewcommand\lstlistingname{Algorithm}
\renewcommand\lstlistlistingname{Algorithms}
\def\lstlistingautorefname{Alg.}

\lstdefinestyle{Python}{
    language        = Python,
    frame           = lines, 
    basicstyle      = \footnotesize,
    keywordstyle    = \color{blue},
    stringstyle     = \color{green},
    commentstyle    = \color{red}\ttfamily
}

\setlength{\parindent}{0.0in}
\setlength{\parskip}{0.05in}

% Edit these as appropriate
\newcommand\course{ConvexOptimization}
\newcommand\hwnumber{3}                  % <-- homework number
\newcommand\NetIDa{SUN Yilin}           % <-- NetID of person #1
\newcommand\NetIDb{520030910361}           % <-- NetID of person #2 (Comment this line out for problem sets)

\pagestyle{fancyplain}
\headheight 35pt
\lhead{\NetIDa}
\lhead{\NetIDa\\\NetIDb}                 % <-- Comment this line out for problem sets (make sure you are person #1)
\chead{\textbf{\Large Homework \hwnumber}}
\rhead{\course \\ \today}
\lfoot{}
\cfoot{}
\rfoot{\small\thepage}
\headsep 1.5em

\begin{document}

\section{}
$\forall\boldsymbol{x}^*,\boldsymbol{y}^*\in M$, consider their convex combination $\theta\boldsymbol{x}^*+\overline{\theta}\boldsymbol{y}^*$. By definition of M, $\boldsymbol{x}^*,\boldsymbol{y}^*\in S$. $S$ is convex, hence $\theta\boldsymbol{x}^*+\overline{\theta}\boldsymbol{y}^*\in S$. Now consider $f(\theta\boldsymbol{x}^*+\overline{\theta}\boldsymbol{y}^*)$. $\forall \boldsymbol{x}\in S,f(\theta\boldsymbol{x}^*+\overline{\theta}\boldsymbol{y}^*)\leq\theta f(\boldsymbol{x}^*)+\overline{\theta}f(\boldsymbol{y}^*)$ by convexity of $f$. Then $\theta f(\boldsymbol{x}^*)+\overline{\theta}f(\boldsymbol{y}^*)\leq \theta f(\boldsymbol{x})+\overline{\theta}f(\boldsymbol{x})=f(\boldsymbol{x}),\forall \boldsymbol{x}\in S$. So we know $\forall \boldsymbol{x}\in S,f(\theta\boldsymbol{x}^*+\overline{\theta}\boldsymbol{y}^*)\leq f(\boldsymbol{x})$, thus $\theta\boldsymbol{x}^*+\overline{\theta}\boldsymbol{y}^*\in M$. So $M$ is convex.

\section{}
Assume $f(\theta_1\boldsymbol{x}+\overline{\theta}_1\boldsymbol{y})<\theta f(\boldsymbol{x})+\overline{\theta}_1 f(\boldsymbol{y})$ for some $\theta_1$. Without loss of generality, assume $\theta_1\in(\theta_0,1]$.\\
$\theta_0\boldsymbol{x}+\overline{\theta}_0\boldsymbol{y}=\frac{\theta_0}{\theta_1}(\theta_1\boldsymbol{x}+\overline{\theta}_1\boldsymbol{y})+(1-\frac{\theta_0}{\theta_1})\boldsymbol{y}$, a convex combination of $\theta_0\boldsymbol{x}+\overline{\theta}_0\boldsymbol{y}$ and $\boldsymbol{y}$. Then by convexity of $f$ we know $f(\theta_0\boldsymbol{x}+\overline{\theta}_0\boldsymbol{y})\leq\frac{\theta_0}{\theta_1}f(\theta_1\boldsymbol{x}+\overline{\theta}_1\boldsymbol{y})+(1-\frac{\theta_0}{\theta_1})f(\boldsymbol{y})<\frac{\theta_0}{\theta_1}(\theta_1f(\boldsymbol{x})+\overline{\theta}_1f(\boldsymbol{y}))+(1-\frac{\theta_0}{\theta_1})f(\boldsymbol{y})=\theta_0f(\boldsymbol{x})+\overline{\theta}_0f(\boldsymbol{y})$. This contradicts with the fact that $f(\theta_0\boldsymbol{x}+\overline{\theta}_0\boldsymbol{y})=\theta_0f(\boldsymbol{x})+\overline{\theta}_0f(\boldsymbol{y})$. Similarly, when $\theta_1\in [0,\theta_0)$, we can also deduce a contradiction. Thus our assumption does not hold. So $f(\theta\boldsymbol{x}+\overline{\theta}\boldsymbol{y})=\theta f(\boldsymbol{x})+\overline{\theta}f(\boldsymbol{y})$ holds for the same $\boldsymbol{x},\boldsymbol{y}$ and any $\theta\in[0,1]$.
\section{}
We solve this problem by calculate the Hessian Matrix.
\subsection*{(a).}
$H=\begin{pmatrix}
    2&0&1\\
    0&2&1\\
    1&1&1
\end{pmatrix}$, the eigenvalues are 3,2,0. Thus $H$ is positive semidefinite, $f$ is convex.
\subsection*{(b).}
$H=\begin{pmatrix}
    \frac{2}{x_1^3x_2}&\frac{1}{x_1^2x_2^2}\\
    \frac{1}{x_1^2x_2^2}&\frac{2}{x_1x_2^3}
\end{pmatrix}$, the leading principle minors are both positive, thus $H$ is positive definite, $f$ is convex.
\subsection*{(c).}
$H=\begin{pmatrix}
    0&2x_2\\
    2x_2&2x_1
\end{pmatrix}$, the principle minors are $0,2x_1,-4x_2^2$. So $H$ is indefinite, $f$ is neither convex nor concave.
\subsection*{(d).}
$H=\begin{pmatrix}
    0&-\frac{1}{2}x_2^{-\frac{3}{2}}\\
    -\frac{1}{2}x_2^{-\frac{3}{2}}&\frac{3}{4}x_1x_2^{-\frac{5}{2}}
\end{pmatrix}$, the principle minors are $0,\frac{3}{4}x_1x_2^{-\frac{5}{2}},-\frac{1}{4}x_2^{-3}$. So $H$ is indefinite, $f$ is neither convex nor concave.
\subsection*{(e).}
$H=\begin{pmatrix}
    \alpha(\alpha-1)x_1^{\alpha-2}x_2^{1-\alpha}&\alpha(1-\alpha)x_1^{\alpha-1}x_2^{-\alpha}\\
    \alpha(1-\alpha)x_1^{\alpha-1}x_2^{-\alpha}&\alpha(\alpha-1)x_1^{\alpha}x_2^{-\alpha-1}
\end{pmatrix}$, the principle minors are $0,\alpha(\alpha-1)x_1^{\alpha-2}x_2^{1-\alpha},\alpha(\alpha-1)x_1^{\alpha}x_2^{-\alpha-1}$. So $H$ is negative semidefinite, $f$ is concave.
\section{}
Let $\boldsymbol{x}=(x_1,x_2),\boldsymbol{y}=(y_1,y_2)$. Then $f(\theta\boldsymbol{x}+\overline{\theta}\boldsymbol{y})=f(\theta x_1+\overline{\theta}y_1,\theta x_2+\overline{\theta}y_2)=f_1(\theta x_1+\overline{\theta}y_1)+f_2(\theta x_2+\overline{\theta}y_2)<\theta f_1(x_1)+\overline{\theta}f_1(y_1)+\theta f_2(x_2)+\overline{\theta}f_2(y_2)=\theta f(\boldsymbol{x})+\overline{\theta}f(\boldsymbol{y})$, so $f$ is strictly convex.\\
When $f(x_1,x_2)=x_1^2+x_2^4$, in this case $f_1$ and $f_2$ are both univariate functions, $f_1''=2>0,f_2''=12x_2^2\geq0$,but $f_2$ is always positive when $x_2\neq0$, so they are both strictly convex, thus $f$ is strictly convex. 
\section{}
Necessity:\\
Since $f$ is a differentiable function with an open convex domain $C$, $f(\boldsymbol{y})\geq f(\boldsymbol{x})+\nabla f(\boldsymbol{x})^T(\boldsymbol{y}-\boldsymbol{x}),\forall \boldsymbol{x},\boldsymbol{y}\in C$, also $f(\boldsymbol{x})\geq f(\boldsymbol{y})+\nabla f(\boldsymbol{y})^T(\boldsymbol{x}-\boldsymbol{y}),\forall \boldsymbol{x},\boldsymbol{y}\in C$. By adding these two inequailities together we get $$0\geq \nabla f(\boldsymbol{x})^T(\boldsymbol{y}-\boldsymbol{x})+\nabla f(\boldsymbol{y})^T(\boldsymbol{x}-\boldsymbol{y}),\forall \boldsymbol{x},\boldsymbol{y}\in C$$ or equivalently $$(\nabla f(\boldsymbol{x})^T-\nabla f(\boldsymbol{y})^T)(\boldsymbol{y}-\boldsymbol{x})\leq 0,\forall \boldsymbol{x},\boldsymbol{y}\in C$$ 
$$(\nabla f(\boldsymbol{x})-\nabla f(\boldsymbol{y}))^T(\boldsymbol{x}-\boldsymbol{y})\geq 0,\forall \boldsymbol{x},\boldsymbol{y}\in C,$$ which is 
$$\langle\nabla f(\boldsymbol{x})-\nabla f(\boldsymbol{y}),\boldsymbol{x}-\boldsymbol{y}\rangle\geq0, \forall \boldsymbol{x},\boldsymbol{y}\in C$$\\
Sufficiency:\\
We prove by showing that for any $\boldsymbol{x}\in C$ and any direction $\boldsymbol{d}$, $g(t)=f(\boldsymbol{x}+t\boldsymbol{d})$ is convex on dom $g$=\{ $t:\boldsymbol{x}+t\boldsymbol{d}\in$ C \}\\
The hint tells us that the intersection of C with a straight line is an open interval when it is not empty, so by our definition dom $g$ is an open interval. Thus to show $g(t)$ is convex, we only need to show $g'(t)$ is increasing.\\
$g'(t)=\boldsymbol{d}^T\nabla f(\boldsymbol{x}+t\boldsymbol{d})$, so $\forall t,s\in $ dom $g$, $(g'(t)-g'(s))(t-s)$
$$=(\boldsymbol{d}^T\nabla f(\boldsymbol{x}+t\boldsymbol{d})-\boldsymbol{d}^T\nabla f(\boldsymbol{x}+s\boldsymbol{d}))(t-s)$$
$$=\boldsymbol{d}^T(\nabla f(\boldsymbol{x}+t\boldsymbol{d})-\nabla f(\boldsymbol{x}+s\boldsymbol{d}))(t-s))$$
$$=(t\boldsymbol{d}^T-s\boldsymbol{d}^T)(\nabla f(\boldsymbol{x}+t\boldsymbol{d})-\nabla f(\boldsymbol{x}+s\boldsymbol{d}))$$
$$=\langle\nabla f(\boldsymbol{x}+t\boldsymbol{d})-\nabla f(\boldsymbol{x}+s\boldsymbol{d}),(t-s)\boldsymbol{d}\rangle$$
But we already have
$$\langle\nabla f(\boldsymbol{x})-\nabla f(\boldsymbol{y}),\boldsymbol{x}-\boldsymbol{y}\rangle\geq0, \forall \boldsymbol{x},\boldsymbol{y}\in C$$
As both $\boldsymbol{x}+t\boldsymbol{d},\boldsymbol{x}+s\boldsymbol{d}\in C$, $\langle\nabla f(\boldsymbol{x}+t\boldsymbol{d})-\nabla f(\boldsymbol{x}+s\boldsymbol{d}),(t-s)\boldsymbol{d}\rangle\geq0$. Which means that $(g'(t)-g'(s))(t-s)\geq 0$. Thus $g'(t)$ is increasing, thus $g(t)$ is convex, by Proposition we have already proven in class, $f$ is convex.
\end{document}
