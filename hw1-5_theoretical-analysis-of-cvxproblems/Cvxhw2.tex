\documentclass[12pt,letterpaper]{article}
\usepackage{fullpage}
\usepackage[top=2cm, bottom=4.5cm, left=2.5cm, right=2.5cm]{geometry}
\usepackage{amsmath,amsthm,amsfonts,amssymb,amscd}
\usepackage{lastpage}
\usepackage{enumerate}
\usepackage{fancyhdr}
\usepackage{mathrsfs}
\usepackage{xcolor}
\usepackage{graphicx}
\usepackage{listings}
\usepackage{hyperref}
\usepackage{mathtools}
\usepackage{xfrac}

\hypersetup{%
  colorlinks=true,
  linkcolor=blue,
  linkbordercolor={0 0 1}
}
\linespread{1.12}
 
\renewcommand\lstlistingname{Algorithm}
\renewcommand\lstlistlistingname{Algorithms}
\def\lstlistingautorefname{Alg.}

\lstdefinestyle{Python}{
    language        = Python,
    frame           = lines, 
    basicstyle      = \footnotesize,
    keywordstyle    = \color{blue},
    stringstyle     = \color{green},
    commentstyle    = \color{red}\ttfamily
}

\setlength{\parindent}{0.0in}
\setlength{\parskip}{0.05in}

% Edit these as appropriate
\newcommand\course{ConvexOptimization}
\newcommand\hwnumber{2}                  % <-- homework number
\newcommand\NetIDa{SUN Yilin}           % <-- NetID of person #1
\newcommand\NetIDb{520030910361}           % <-- NetID of person #2 (Comment this line out for problem sets)

\pagestyle{fancyplain}
\headheight 35pt
\lhead{\NetIDa}
\lhead{\NetIDa\\\NetIDb}                 % <-- Comment this line out for problem sets (make sure you are person #1)
\chead{\textbf{\Large Homework \hwnumber}}
\rhead{\course \\ \today}
\lfoot{}
\cfoot{}
\rfoot{\small\thepage}
\headsep 1.5em

\begin{document}

\section{}
This problem requires us to show that $f^{-1}(C)$ is convex. We want to show that if $\boldsymbol{x_{1}},$ $\boldsymbol{x_{2}} \in f^{-1}(C)$, then $\theta \boldsymbol{x_{1}}+\overline{\theta}\boldsymbol{x_{2}} \in f^{-1}(C)$ for any $\theta \in [0,1]$. To show whether $\theta \boldsymbol{x_{1}}+\overline{\theta}\boldsymbol{x_{2}} \in C$ or not, we can check $f(\theta \boldsymbol{x_{1}}+\overline{\theta}\boldsymbol{x_{2}})$.\\
$f(\theta \boldsymbol{x_{1}}+\overline{\theta}\boldsymbol{x_{2}})=\boldsymbol{A}(\theta \boldsymbol{x_{1}}+\overline{\theta}\boldsymbol{x_{2}})+\boldsymbol{b}=\boldsymbol{A}\theta \boldsymbol{x_{1}}+\boldsymbol{A}\overline{\theta}\boldsymbol{x_{2}}+\theta\boldsymbol{b}+\overline{\theta}\boldsymbol{b}=\theta(\boldsymbol{Ax_{1}}+\boldsymbol{b})+\overline{\theta}(\boldsymbol{Ax_{2}}+\boldsymbol{b})=\theta f(\boldsymbol{x_{1}})+\overline{\theta}f(\boldsymbol{x_{2}})$. Since $\boldsymbol{x_{1}},$ $\boldsymbol{x_{2}} \in f^{-1}(C)$, $f(\boldsymbol{x_{1}}),$ $f(\boldsymbol{x_{2}}) \in C$, then by the convexity of $C$, $\theta f(\boldsymbol{x_{1}})+\overline{\theta}f(\boldsymbol{x_{2}})\in C$, $i.e.$ $f(\theta \boldsymbol{x_{1}}+\overline{\theta}\boldsymbol{x_{2}})\in C$. Thus $\theta \boldsymbol{x_{1}}+\overline{\theta}\boldsymbol{x_{2}} \in f^{-1}(C)$ for any $\theta \in [0,1]$. Thus $f^{-1}(C)$ is convex.
\section{}
Since $C_{1}$ and $C_{2}$ are nonempty, $\exists \boldsymbol{x_{1}}\in C_{1}$, $\exists \boldsymbol{x_{2}}\in C_{2}$. Thus by definition of $C$, $\boldsymbol{x_{1}}-\boldsymbol{x_{2}}\in C$. Thus $C$ is nonempty.\\
If $\boldsymbol{0}\in C$, then $\exists \boldsymbol{x_{1}}\in C_{1}$, $\exists \boldsymbol{x_{2}}\in C_{2}$ such that $\boldsymbol{{x}_{1}}=\boldsymbol{{x}_{2}}$, then $C_{1}\cap C_{2}\neq \varnothing$, which contradicts the fact that $C_{1}\cap C_{2}=\varnothing$. Thus $\boldsymbol{0}\notin C$.\\
Assume $\boldsymbol{x},\boldsymbol{y}\in C$, then $\exists \boldsymbol{x_{1}},\boldsymbol{y_{1}}\in C_{1}$, $\exists \boldsymbol{x_{2}},\boldsymbol{y_{2}}\in C_{2}$ such that $\boldsymbol{x}=\boldsymbol{x_{1}}-\boldsymbol{x_{2}},\boldsymbol{y}=\boldsymbol{y_{1}}-\boldsymbol{y_{2}}$. Consider $\theta \boldsymbol{x}+\overline{\theta}\boldsymbol{y}$ where $\theta\in [0,1]$, $\theta \boldsymbol{x}+\overline{\theta}\boldsymbol{y}=\theta(\boldsymbol{x_{1}}-\boldsymbol{x_{2}})+\overline{\theta}(\boldsymbol{y_{1}}-\boldsymbol{y_{2}})=(\theta\boldsymbol{x_{1}}+\overline{\theta}\boldsymbol{y_{1}})-(\theta\boldsymbol{x_{2}}+\overline{\theta}\boldsymbol{y_{2}})$. By convexity of $C_{1}$ and $C_{2}$ we know $\theta\boldsymbol{x_{1}}+\overline{\theta}\boldsymbol{y_{1}}\in C_{1}$ and $\theta\boldsymbol{x_{2}}+\overline{\theta}\boldsymbol{y_{2}}\in C_{2}$, thus $\exists \theta\boldsymbol{x_{1}}+\overline{\theta}\boldsymbol{y_{1}}\in C_{1}$, $\exists \theta\boldsymbol{x_{2}}+\overline{\theta}\boldsymbol{y_{2}}\in C_{2}$ such that $\theta \boldsymbol{x}+\overline{\theta}\boldsymbol{y}=(\theta\boldsymbol{x_{1}}+\overline{\theta}\boldsymbol{y_{1}})-(\theta\boldsymbol{x_{2}}+\overline{\theta}\boldsymbol{y_{2}})$, which means $\theta \boldsymbol{x}+\overline{\theta}\boldsymbol{y}\in C$. So $C$ is convex.

\section{}
\subsection*{(a)}
We prove this by contradiction.\\
Assume $intC$ is not convex. Then $\exists\boldsymbol{x}_0,\boldsymbol{y}_0\in intC$, $\exists \theta_0\in(0,1)$ such that $\boldsymbol{z}_0=\theta_0\boldsymbol{x}_0+\overline{\theta_0}\boldsymbol{y}_0\notin intC$. Then if we fix $\theta_0$ and $\boldsymbol{x}_0$ and treat $\boldsymbol{z}=\theta_0\boldsymbol{x}_0+\overline{\theta_0}\boldsymbol{y}$ as a  continuous function, given a neighbourhood $Y\subset C$ of $\boldsymbol{y}_0$, there exists a neighbourhood $Z$ of $\boldsymbol{z}_0$ such that for all points in $Z$ the corresponding $y$ is in $Y$. Since $\boldsymbol{z}_0\notin intC$, there exists $\boldsymbol{z}_1\in Z$ but $\boldsymbol{z}_1\notin C$. Let $\boldsymbol{y}_1$ be the corresponding $\boldsymbol{y}$ for that $\boldsymbol{z}_1$, then $\boldsymbol{z}_1=\theta_0\boldsymbol{x}_0+\overline{\theta_0}\boldsymbol{y}_1$ where $\boldsymbol{x}_0,\boldsymbol{y}_1\in C$. But by our assumption we conclude that their convex combination $\boldsymbol{z}_1\notin C$, this means that our assumption that $intC$ is not convex contradicts with the fact that $C$ is convex. Thus $intC$ is also convex. 
\subsection*{(b)}
We prove this directly.\\
$\forall\boldsymbol{x},\boldsymbol{y}\in\overline{C}$, for any $r>0$, $\exists\boldsymbol{x}_0,\boldsymbol{y}_0\in C$ such that $||\boldsymbol{x}-\boldsymbol{x}_0||<r,||\boldsymbol{y}-\boldsymbol{y}_0||<r$. Thus for any convex combination $\theta\boldsymbol{x}+\overline{\theta}\boldsymbol{y},\theta\in(0,1)$ and any $r>0$, $||(\theta\boldsymbol{x}+\overline{\theta}\boldsymbol{y})-(\theta\boldsymbol{x}_0+\overline{\theta}\boldsymbol{y}_0)||=||(\theta\boldsymbol{x}-\theta\boldsymbol{x}_0)+(\overline{\theta}\boldsymbol{y}-\overline{\theta}\boldsymbol{y}_0)||\leq||\theta\boldsymbol{x}-\theta\boldsymbol{x}_0||+||\overline{\theta}\boldsymbol{y}-\overline{\theta}\boldsymbol{y}_0||<r$. By convexity of $C$ we know $\theta\boldsymbol{x}_0+\overline{\theta}\boldsymbol{y}_0\in C$, thus any  $\theta\boldsymbol{x}+\overline{\theta}\boldsymbol{y},\theta\in(0,1)$ has a $\theta\boldsymbol{x}_0+\overline{\theta}\boldsymbol{y}_0\in C$ in any of its neighbourhood. This means that $\theta\boldsymbol{x}+\overline{\theta}\boldsymbol{y}\in\overline{C}$. So $\overline{C}$ is convex.
\section{}
\subsection*{(a)}
Let $\boldsymbol{x},\boldsymbol{y}\in C$, where $\boldsymbol{x}=\sum_{i=1}^{m}\theta_{i}\boldsymbol{x}_i;$ $\boldsymbol{x}_i\in S, \theta_i\geq0,i=1,\dots,m;\sum_{i=1}^{m}\theta_i=1$ and $\boldsymbol{y}=\sum_{i=1}^{n}\phi_{i}\boldsymbol{y}_i;$ $\boldsymbol{y}_i\in S, \phi_i\geq0,i=1,\dots,n;\sum_{i=1}^{n}\phi_i=1$. Then we check $\alpha\boldsymbol{x}+\overline{\alpha}\boldsymbol{y}$ where $\alpha \in [0,1]$\\
$\alpha\boldsymbol{x}+\overline{\alpha}\boldsymbol{y}=\alpha \sum_{i=1}^{m}\theta_{i}\boldsymbol{x}_i+\overline{\alpha}\sum_{j=1}^{n}\phi_{j}\boldsymbol{y}_j= \sum_{i=1}^{m}\alpha\theta_{i}\boldsymbol{x}_i+\sum_{j=1}^{n}\overline{\alpha}\phi_{j}\boldsymbol{y}_j$. In this expression $m+n\in \mathbb{N}; \boldsymbol{x}_i,\boldsymbol{y}_j\in S,\alpha\theta_i\geq0,\overline{\alpha}\phi_{j}\geq0,i=1,\dots,m,j=1,\dots,n$. Also $\sum_{i=1}^{m}\alpha\theta_i+\sum_{j=1}^{n}\overline{\alpha}\phi_{j}=\alpha\sum_{i=1}^{m}\theta_i+\overline{\alpha}\sum_{j=1}^{n}\phi_{j}=\alpha+\overline{\alpha}=1$. Thus $\alpha\boldsymbol{x}+\overline{\alpha}\boldsymbol{y}\in C$, so $C$ is convex.
\subsection*{(b)}
$convS$ is the smallest convex set containing S, thus $S\subset convS$. Thus $\forall\boldsymbol{x}\in S$, $\boldsymbol{x}\in convS$. Since $convS$ is convex, $\forall\boldsymbol{x}_i \in convS,i=1,\dots,m$, their convex combination $\sum_{i=1}^{m}\theta_{i}\boldsymbol{x}_i\in convS$ by the theorem we have proved in class. By definition of $C$ we know $\forall\boldsymbol{x}\in C$, $\boldsymbol{x}=\sum_{i=1}^{m}\theta_{i}\boldsymbol{x}_i$ where $\boldsymbol{x}_i\in convS$ for $i=1.\dots,m$ and $\sum_{i=1}^{m}\theta_{i}=1$. Thus $\boldsymbol{x}$ is a convex combination of elements in $convS$. This means that any element of $C$ is a convex combination of elements in $convS$. We have shown above that the convex combination of elements in $convS$ must also be an element in $convS$(by its convexity), thus we can say $\forall\boldsymbol{x}\in C$, $\boldsymbol{x}\in convS$, $i.e.$ $C\subset convS$. And by definition of $convS$ we know $convS\subset C$, thus $C=convS$.
\section{}
$\forall\boldsymbol{x}\in\boldsymbol{V}$, $||\boldsymbol{x}-\boldsymbol{x_{0}}||_{2}\leq||\boldsymbol{x}-\boldsymbol{x_{i}}||_{2},i=1,2,\dots,K$. In $\mathbb{R}^n$ this can be written as $$(\boldsymbol{x}-\boldsymbol{x}_0)^{T}(\boldsymbol{x}-\boldsymbol{x}_0)\leq(\boldsymbol{x}-\boldsymbol{x}_i)^{T}(\boldsymbol{x}-\boldsymbol{x}_i)$$
$$(\boldsymbol{x}^T-\boldsymbol{x}^T_0)(\boldsymbol{x}-\boldsymbol{x}_0)\leq(\boldsymbol{x}^T-\boldsymbol{x}^T_i)(\boldsymbol{x}-\boldsymbol{x}_i)$$
$$\boldsymbol{x}^T\boldsymbol{x}-2\boldsymbol{x}^T_0\boldsymbol{x}+\boldsymbol{x}^T_0\boldsymbol{x}_0\leq\boldsymbol{x}^T\boldsymbol{x}-2\boldsymbol{x}^T_i\boldsymbol{x}+\boldsymbol{x}^T_i\boldsymbol{x}_i$$
$$2(\boldsymbol{x}_i^T-\boldsymbol{x}_0^T)\boldsymbol{x}\leq\boldsymbol{x}_i^T\boldsymbol{x}_i-\boldsymbol{x}_0^T\boldsymbol{x}_0$$\\
That inequality holds for any $i=1,2,\dots,K$. Then by writing the inequalities in matrix form we get 
$$
\begin{pmatrix}
2\boldsymbol{x}_1^T-2\boldsymbol{x}_0^T\\
2\boldsymbol{x}_2^T-2\boldsymbol{x}_0^T\\
\dots\\
2\boldsymbol{x}_K^T-2\boldsymbol{x}_0^T
\end{pmatrix}
\boldsymbol{x}\leq
\begin{pmatrix}
\boldsymbol{x}_1^T\boldsymbol{x}_1-\boldsymbol{x}_0^T\boldsymbol{x}_0\\
\boldsymbol{x}_2^T\boldsymbol{x}_2-\boldsymbol{x}_0^T\boldsymbol{x}_0\\
\dots\\
\boldsymbol{x}_K^T\boldsymbol{x}_K-\boldsymbol{x}_0^T\boldsymbol{x}_0
\end{pmatrix}
$$
Let 
$$
\boldsymbol{A}=\begin{pmatrix}
2\boldsymbol{x}_1^T-2\boldsymbol{x}_0^T\\
2\boldsymbol{x}_2^T-2\boldsymbol{x}_0^T\\
\dots\\
2\boldsymbol{x}_K^T-2\boldsymbol{x}_0^T
\end{pmatrix},\boldsymbol{b}=\begin{pmatrix}
\boldsymbol{x}_1^T\boldsymbol{x}_1-\boldsymbol{x}_0^T\boldsymbol{x}_0\\
\boldsymbol{x}_2^T\boldsymbol{x}_2-\boldsymbol{x}_0^T\boldsymbol{x}_0\\
\dots\\
\boldsymbol{x}_K^T\boldsymbol{x}_K-\boldsymbol{x}_0^T\boldsymbol{x}_0
\end{pmatrix}
$$
we know $V=\{\boldsymbol{x}:\boldsymbol{Ax}\leq\boldsymbol{b}\}$. Thus $V$ is a polyhedron.

\end{document}
