\documentclass[12pt,letterpaper]{article}
\usepackage{fullpage}
\usepackage[top=2cm, bottom=4.5cm, left=2.5cm, right=2.5cm]{geometry}
\usepackage{amsmath,amsthm,amsfonts,amssymb,amscd}
\usepackage{lastpage}
\usepackage{enumerate}
\usepackage{fancyhdr}
\usepackage{mathrsfs}
\usepackage{xcolor}
\usepackage{graphicx}
\usepackage{listings}
\usepackage{hyperref}
\usepackage{mathtools}
\usepackage{xfrac}
\usepackage{bbm}

\hypersetup{
  colorlinks=true,
  linkcolor=blue,
  linkbordercolor={0 0 1}
}
\linespread{1.1}
 
\renewcommand\lstlistingname{Algorithm}
\renewcommand\lstlistlistingname{Algorithms}
\def\lstlistingautorefname{Alg.}


\lstdefinestyle{Python}{
    language        = Python,
    frame           = lines, 
    basicstyle      = \footnotesize,
    keywordstyle    = \color{blue},
    stringstyle     = \color{green},
    commentstyle    = \color{red}\ttfamily
}

\setlength{\parindent}{0.0in}
\setlength{\parskip}{0.05in}

\newcommand\course{Computer Vision}
\newcommand\hwnumber{1}
\newcommand\NetIDa{SUN Yilin}
\newcommand\NetIDb{520030910361}

\pagestyle{fancyplain}
\headheight 35pt
\lhead{\NetIDa}
\lhead{\NetIDa\\\NetIDb}
\chead{\textbf{\Large Homework \hwnumber}}
\rhead{\course \\ \today}
\lfoot{}
\cfoot{}
\rfoot{\small\thepage}
\headsep 1.5em

\begin{document}

\section{Written Assignment}
\subsection*{a}
The shape of the image will still be a circular disk, with different radius.\\
\textbf{Explanations:}
We use the same setting as our lecture notes, i.e. $\boldsymbol{r_o}=(x_o,y_o,z_o)$ denotes
the coordinate of actual object pixel and $\boldsymbol{r_i}=(x_i,y_i,f)$ denotes its image.
We set the pinhole $(x,y,z)=(0,0,0)$ and let the optical axis be the $z$-axis.
Now we prove our argument by deriving the equation of the image of the disk.\\
The equation of the circular disk in original space can be described as $(x_o-a)^2+(y_o-b)^2=c^2$
where $(a,b)$ is its center and $c$ is its radius. Then by applying the rule of similar triangles 
we get $(x_i,y_i)=(\frac{fx_o}{z_o},\frac{fy_o}{z_o})$, then by substituting the coordinates
we get the equation of image disk as $(\frac{z_ox_i}{f}-a)^2+(\frac{z_oy_i}{f}-b)^2=c^2$,
or equivalently $$(x_i-\frac{af}{z_o})^2+(y_i-\frac{bf}{z_o})=(\frac{cf}{z_o})^2$$
which is still a circular disk in image plane.
\subsection*{b}
\textbf{Case 1:} plane $y=0$\\
The direction vectors on this plane can be expressed generally as $(x_i,0,z_i)$.
Then the vanishing points of the family of lines represented by these family of vectors
can be expressed as $(f\frac{x_i}{z_i},0)$ by our formula from lecture slides.
Then we know the vanishing points $(f\frac{x_i}{z_i})$ actual lie on line $y=0$ in the image plane.\\
\textbf{Case 2:} plane $x=0$\\
Now the direction vectors are $(0,y_i,z_i)$ and the vanishing points are $(0,f\frac{y_i}{z_i})$.
This time the vanishing points lie on the line $x=0$ in the image plane.
\subsection*{c}
Generally, the normal vector of plane $Ax+By+Cz+D=0$ is $(A,B,C)$, thus any direction vector $(x_i,y_i,z_i)$
that lies in this plane must satisfy that $Ax_i+By_i+Cz_i=0$. 
Then we know that the vanishing point of line $(x_i,y_i,z_i)$ is $(f\frac{x_i}{z_i},f\frac{y_i}{z_i})$.
(Note that $z_i$ cannot be zero, otherwise there will not be a vanishing point for these lines parallel to the image plane)\\
These vanishing points $(f\frac{x_i}{z_i},f\frac{y_i}{z_i})$ actually lie on line
$$Ax+By+Cf=0$$

\newpage
\section{Programming Assignment Documentation and Results}
\end{document}